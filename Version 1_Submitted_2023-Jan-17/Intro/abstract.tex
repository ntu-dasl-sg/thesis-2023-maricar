\theabstract{

Amidst increasing intensity of hazards, population growth, and often unplanned urbanisation trends, it is encouraging that the past few decades have seen significant development in state-of-art tools that quantify risk from natural hazards. These frameworks supply essential information to risk reduction managers, enabling them to make informed decisions towards promoting greater resilience. However, current frameworks still under-emphasise elements important for decision-making which include (1) considering spatial and uncertainty characteristics in hazard modelling, (2) incorporating time-dependent processes that affect vulnerability, and (3) highlighting the successes and benefits of risk reduction. The main contributions of this dissertation include a framework for hazard modelling using limited and uncertain spatial data, a framework for modelling time-dependent vulnerability in regional risk analysis, and a framework to incentivise and learn from effective risk reduction. 
Altogether, they shift the field of risk and hazard quantification towards analytics that can better support decision-making in reducing the risk of constantly evolving regions.


% (ASE Requirement) The Abstract or Summary should summarise the appropriate headings, aims, scope and conclusion of the thesis. It should be of approximately 150 words.
% Your abstract should capture, in English, the whole thesis with focus on the problem and solution in 150 words. It should be placed on a separate right-hand page, with an additional \textit{1cm} margin on both left and right. Avoid acronyms, footnotes, and references in the abstract if possible.

% Leave a \textit{2cm} vertical space after the abstract and provide a few keywords relevant for your report. Use five to six words, of which at most two should be from the title.

% What is the reason for writing the thesis?
% What are the current approaches and gaps in the literature? What are your research question(s) and aims?
% Which methodology have you used?
% What are the main findings?
% What are the main conclusions and implications?

% Resources:
% https://www.thephdproofreaders.com/writing/how-to-write-an-abstract-for-your-phd-thesis/ 


}
\keywords{
probabilistic risk analysis;
spatio-temporal processes;
counterfactual analysis;
model calibration;
earthquake risk;
tephra fall hazard;
school earthquake safety}

% Tephra; Inversion; Model calibration; Kriging
% Performance-Based Earthquake Engineering; Risk analysis; Markov chains; 
% Probabilistic risk, Disaster risk reduction, Risk framework, 