\acknowledgements
{

This dissertation is supported mainly by Singapore's National Research Foundation (NRF) under the NRF-NRFF2018-06 award, the Ministry of Education (MOE), and Nanyang Technological University (NTU) as a Research Centre of Excellence. I also acknowledge the additional support and scholarship funding from the Earth Observatory of Singapore (EOS). I describe individual project contributions in a dedicated Authorship Attribution Statement. Project-specific acknowledgements and funding are included at the end of each chapter.

This PhD journey would not have been possible without the support of countless people around the world. I have been privileged to work in the most supportive, fun, and intellectually vibrant environment. Here, I briefly describe those who have assisted me professionally and emotionally. 

To my adviser, David Lallemant, thank you for being a constant source of support and positive energy since day one of my PhD. My research process has been extremely rewarding because of the freedom, unwavering guidance, and enthusiasm that you provided. I learned from you how to become a more effective, reflective, and intentional researcher. I am confident to say that you have helped me build the most amazing PhD. You have my deep gratitude.

I want to thank my esteemed thesis advisory committee members, Susanna Jenkins and Janice Teresa Ser Huay Lee, who have been a limitless source of wisdom and generosity over the last few years. Thank you to Susanna for the opportunity to collaborate with the volcano group in the Asian School of the environment, which lead to a tremendous learning experience. The learning curve for me have been steep, but you made sure I got the support I needed on the volcanology aspects, leading to one of the most rewarding research work I've done in the PhD - thank you. I am grateful to Janice for generously giving me advice on technical and non-technical aspects of being a researcher in the field. I learned how to write my first policy paper from your teaching, which made me realise new ways of thinking about success in risk reduction. Special thanks to Benoit Taisne who is generous with his time to join my thesis committee meetings while making sure he asks the tough (but super helpful) questions. Susanna, Janice, and Benoit, your patient, careful mentorship and engagement with my ideas has always made my commitment stronger and my work better. 

I am thankful to the incredible training by some of the most exceptional faculty and instructors. Beyond those previously mentioned, I express my thanks to Emma Hill, Sang-Ho Yun, Adam Switzer, Patrick Martin, Pavel Adamek, Karen Lythgoe, Lujia Feng, and Eric Lindsey. To the editors and reviewers of my published work, I express my utmost appreciation for the thoughtfulness, care and professionalism you’ve put into the peer-review process. Thank you for the time spent suggesting improvements, engaging with, and reviewing my research, Gemma Cremen, Carmine Galasso, Gabriele Fiorentino, JC Gaillard, and Martin Joe.

I have worked, been mentored, and developed friendships with very brilliant researchers and professors. Yolanda Lin, the first few years of my PhD has been transformative for me as a researcher, and I want to express my gratitude to you for being a core part of it. You have been a personal inspiration in terms of the standards of work ethics that you demonstrate, and I am so appreciative of your support as a colleague and as a friend. Michele Nguyen, thank you for the generous support as I bridge boundaries and scientific methods in my research. I am grateful of your thoughtfulness, careful attention to detail, and insightful mentorship. Sebastien Biasse, my research on tephra inversion is much better because of your generosity in sharing your expertise. I highly appreciate your words of enthusiasm towards my ideas, as well as the patience and effort to explain the concepts of volcanic eruption parameters.  Gizem Mestav Sarica, thank you for our friendship and how we continued to support one another in our work on urban growth models and beyond our academic career. Poul Grashoff, thank you for training me to use the Java Spatial Model (urban growth model) and spending hours to compile the results so that we can explore its potential for zoning policies. Daniel Vaulot, working alongside with you to teach the Data Science class has been a delightful and important learning experience for me in terms of effective teaching. 
% Your digital drawings of how tephra disperses and deposits were one of the most useful drawings for my tephra research.

My labmates in the Disaster Analytics for Society Lab (DASL) deserve a shoutout. I'm grateful to have been part of a community that develops tools and analysis in the field of natural hazards in a way that's respectful and mindful of the implications to society. Yolanda, Feroz, Tian Ning, Sanjana, Michele, Gizem, Sabine, Mariano, Dennis, Neel, Alina, Jeanette, Celine, Pouria, Sonali, and Marcus, you all have made my PhD experience fun and intellectually engaging. Thank you as well for being amazing colleagues for our work on the Global Assessment Report paper and modelling for Typhoon Goni impacts. Those are some intense crunch times. Alina, thank you for the thorough review of my thesis Introduction. I am grateful to you, Jeanette, and Dennis for reviewing Chapter 1 given my one-day notice.

I am thankful for the chance to be connected to amazing people and researchers who provided me data and logistical resources for my research. Thank you for being thoughtful and clear in the formats and meaning of the data you provided: Nama Budhathoki (Kathmandu Living Labs), Shengji Wei, Meng Chen (Earth Observatory of Singapore), George Williams, Jasna Budhathoki, and Victoria Stevens. Christina Tee Siew Khiaw and Nuraishah Binte Kasmadi, you are rockstars in terms of the administrative support that you provide. Edwin Tan, your instructions and support have been invaluable in my use of the Komodo cluster for tephra inversion.

% I am grateful for Jasna Budhathoki for the time and effort she spent browsing for more useful information about the schools retrofitting in Nepal. I express my thanks to Victoria Stevens for her detailed explanations about her work on the probabilistic seismic hazard assessment for Nepal.

I have also been lucky enough to be connected and have fruitful discussions with experts passionate in methods for dynamic risk. Their work has inspired my thinking on the topic, and broaden my knowledge on the implications and capabilities of dynamic risk modelling. Thank you to Rashmin Gunasekera, Anirudh Rao, James Daniell, Antonios Pomonis, Graeme Riddell, Gizem Sarica, and Hedwig van Delden.

It is pure enjoyment to host visiting scholars in NTU. I am thankful for the opportunity to personally interact (even though brief) with scholars whose work I found most insightful: David Wald, Brian McAdoo, JC Gaillard, Carlos Molina-Hutt, Dale Dominey-Howes, Gordon Woo, Guillermo Franco, and Olivia Jensen.

%%% ADA Team
Probably one of the most unexpected yet impactful experience in my PhD is contributing to the design and launch of the Averted Disaster Award (ADA). My utmost thanks to support of the Understanding Risk Community, Global Facility for Disaster Reduction and Recovery (GFDRR), and the World Bank Group for its launch in Singapore and award ceremony in Brazil. Thank you for funding my trip to Brazil so that we can announce the first ever winners of the Averted Disaster Award (fun!). Thank you to Francis Ghesquiere for his support and enthusiasm, which made the idea start to materialise. I'm very lucky to have worked with the amazing ADA team: Carrie Levine, James Martin Ennis, Alan Dinca, Pablo Suarez, Kara Devonna Siahann, and of course David Lallemant. Thank you for the immense and intricate work it took to build such a prestigious award. Witnessing how an idea that I worked on in the PhD produced a systematic global recognition of effective risk reduction makes me the happiest.

Much of the work I presented in this thesis are technical, but what isn't shown are the countless social media posts, blog posts, newsletters, and interviews that helped share the concepts of my work to the general audience. For this I had much help, in which I am truly grateful. Thank you for providing your media, risk communication, and/or photography expertise so that I can share the counterfactual analysis work to the public effectively: Lauriane Chardot, Mark Zuckerman, Beh Lih Yi, Nick Paul, and Jared Ng.

I have found amazing community from my time in Singapore, as well as during my month-long stay in Chiang Mai, Thailand.  Special shoutout to Gina Sarkawi and Anushka Rege, for being the most amazing housemates, and who are always the first people I know I can count on in Singapore. I will always remember my PhD journey with a smile because of the goofy, intellectually-stimulating, and sometimes shocking experiences that we shared together. To my friends in Singapore - Wardah, Joanne, Tian Ning, Gizem, Sanjana, Hanna, Harris, Amit, Lesley, Jessica Cho - I have so much gratitude for your warmth and kindness. Because of the Chiang Mai Field Lab, I think the number of friends I have easily increased by 30+. Specifically, I'd like to thank Ashrika Sharma, Karen Barns, Pamela Cajilig, Giuseppe Molinario, Kei Franklin, Perrine Hamel, and Robert Soden for keeping the friendly connections since then. To my lovely friends in Manila: Audrei and Rich Yba\~{n}ez, Gel, Julisa, and \textit{"103"} buddies, thank you for being there for me and the consistent friendship.

%% Parents
I want to thank my family for providing the resources that have allowed me to pursue this path. \textit{Salamat po, mommy, daddy, ate, kuya bibi, at kuya manong.} I think my brother, Rian, would have been proud, had he lived to see this day. Mom, Nenita L. Rabonza, I may be the first PhD in our whole extended family, but you know, you are still the most intelligent and most resilient person I know. Thank you for making sure that you’re taking best care of yourself and dad, so that I never have to worry, and have the mindspace to take on my endeavors. This wouldn’t be possible without you. Dad, Manuel Rabonza, thank you for always providing me the wisest advice I need most at every moment. You taught me the value of routine, hardwork, and calmness (amidst chaos). You and mom are my rock. Cherry Rabonza, my \textit{ate} (sister), you have always been a source of warmth and safety in my life. The house we bought and renovated together in the last few years has been an exciting project that always inspired me to work harder. %kuya bibi

And Inah, thank you for believing in me even when I hadn’t yet. The pandemic has been a harsh time for me, being away with people I love most, but you kept me sharp, kept me sane, and made sure these pages are written. Every paper I wrote, you’ve transformed to the most exciting stories over countless dinners with family and friends. The safety of your love kept me going. Thank you.

In 2013, just six months after I started working on analytics of natural hazards, I witnessed first-hand the devastation of Typhoon Haiyan in the Philippines. I dedicate this dissertation to those who have survived such disaster events and others who work tirelessly to minimise the impacts from future disasters.



% Each of you have given your energy, time, and I am richer for it:
% constant source of inspiration
% If you want to thank people, do it here, on a separate right-hand page. Both the U.S. \textit{acknowledgments} and the British \textit{acknowledgements} spellings are acceptable.

% If you do not want to acknowledge anyone, just remove this text or delete the acknowledgements.tex file.
}
