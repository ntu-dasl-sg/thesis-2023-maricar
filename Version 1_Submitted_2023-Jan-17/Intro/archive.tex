Author contribution:



% \\ \\ \noindent
% Included in the Annex and cited in Chapter 2 as Supplementary Material is following conference paper:
% \\ \\ \noindent
% \textbf{Rabonza, M.L.} and Lallemant, D. (2018). A time-dependent model for seismic risk reduction policy analysis. Accepted in the \textit{17th U.S.-Japan-New Zealand Workshop on the Improvement of Structural Engineering and Resilience Nov 2018}

% Chapter 1 is the combination of 2 published (peer-reviewed) conference papers,  and 1 journal paper in preparation. 
%     \\ \\ \noindent
%     The first conference paper is published as:
%     \\
%     \textbf{Rabonza, M.L.} and Lallemant, D. (2019). Accounting for time and state-dependent vulnerability of structural systems. \textit{In Proceedings of the 13th International Conference on Applications of Statistics and Probability in Civil Engineering (ICASP13) Sep 2019}. https://doi.org/10.22725/ICASP13.465
%     \\ \\ \noindent
%     The second conference paper is published as:
%     \\ 
%     \textbf{Rabonza, M.L.} and Lallemant, D. (2018). A time-dependent model for seismic risk reduction policy analysis. Accepted in the \textit{17th U.S.-Japan-New Zealand Workshop on the Improvement of Structural Engineering and Resilience Nov 2018}
%     \\ \\ \noindent
%     Parts of Chapter 1 that focus on \textit{changing exposure/urban growth} is a journal manuscript in preparation:
%     \\ 
%     \textbf{Rabonza, M.L.} and Lallemant, D. (2023). A time-dependent urban risk model for policy analysis and risk targeting in disaster risk reduction. \textit{Journal manuscript In-prep}
%     \begin{itemize}
%     \item For all the papers above, I designed the study with David Lallemant. I collected the data, performed the analysis and wrote the paper.
%     \end{itemize}



% \subsubsection*{Chapter 3:}
% \\ \\ \noindent
% Chapter 3 is the combination of 1 published journal article, 1 published contributing paper to a United Nations report (also under review in a journal), and 1 peer-reviewed conference paper.
%     \\ \\ \noindent
%     The journal article is published as:
%     \\ 
%     \textbf{Rabonza M.L.}, Lin Y.C. and Lallemant D. (2022) Learning From Success, Not Catastrophe: Using Counterfactual Analysis to Highlight Successful Disaster Risk Reduction Interventions. \textit{Front. Earth Sci.} 10:847196. doi: 10.3389/feart.2022.847196

%     \begin{itemize}
%     \setlength\itemsep{-0.6em}
%     \item I lead the writing, research and analysis of both case studies. 
%     \item David Lallemant conceived of the idea of celebrating successes in disaster risk reduction using counterfactual analysis. 
%     \item Yolanda Lin and David Lallemant provided critical feedback that shaped the research, analysis and manuscript.
%     \item All authors contributed to the conceptualisation and design of the study. 
%     \end{itemize}

%     \\ \\ \noindent
%     Parts of the chapter focused on invisibilities of risk reduction is a paper developed as a Contributing Paper to the Global Assessment Report (GAR) on Disaster Risk Reduction 2022 by the United Nations Office for Disaster Risk Reduction (UNDRR). The paper is also under review in the International Journal of Disaster Risk Reduction journal (IJDRR) for publication as an original research article. The manuscript is published as:
%     \\ \\
%     \textbf{Rabonza, M.L.}, Lallemant,  D.,  Lin,  Y. C., Tadepalli,  S., Wagenaar,  D., Nguyen,  M., Choong,  J., Liu,  C. J. N., Sarica,  G. M., Widawati,  B. A. M., Balbi,  M., Khan,  F., Loos,  S. \& Lim,  T. N. (2022). Shedding light on avoided disasters : measuring the invisible benefits of disaster risk management using probabilistic counterfactual analysis. \textit{A contributing paper to the United Nations Office for Disaster Risk Reduction (UNDRR) Global Assessment Report 2022}. https://www.undrr.org/publication/shedding-light-avoided-disasters-measuring-invisible-benefits-disaster-risk-reduction
%     \begin{itemize}
%     \setlength\itemsep{-0.6em}
%     \item I lead the writing, analysis design, and conceptualisation of the invisibilities in mitigation successes with David Lallemant.
%     \item I conceptualised the use of counterfactual analysis to address the invisibilities in risk reduction success with Yolanda Lin and David Lallemant.
%     \item I produced the schematic figure for the types of invisibilities.
%     \item For the risk perception section, Sanjana Tadepalli and Feroz Khan provided more references, and assisted with editing.
%     \item For the risk analysis framework section, Mariano Balbi and Yolanda Lin provided feedback on the writing.
%     \item For the Nepal case study, I performed the analysis, and Yolanda Lin and Sabine Loos provided feedback on the writing.
%     \item Michele Nguyen and Gizem Mestav Sarica assisted in collecting examples of DRM measures for which counterfactual analysis can be applied.
%     \item All authors discussed the results and commented on the manuscript.
%     \end{itemize}
%     \\ \\ \noindent
%     Parts of the chapter focused on the Nepal case study was published as a peer-reviewed conference paper:
%     \\ \\
%     \textbf{Rabonza, M.L.}, Lin, Y., Lallemant, D. (2020, September). Celebrating successful earthquake risk reduction through counterfactual probabilistic analysis. \textit{In Proceedings of the 17th World Conference on Earthquake Engineering (Article No. 8c-0059) Sendai, Japan}. https://hdl.handle.net/10356/152837
%     \begin{itemize}
%     \setlength\itemsep{-0.5em}
%     \item I designed the research with Yolanda Lin and David Lallemant. 
%     \item I performed the analysis and wrote the manuscript. 
%     \end{itemize}
