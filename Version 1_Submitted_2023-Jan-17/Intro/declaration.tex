\declaration{

%%%%%%%%%%%%%%%%%%%%%%%%%
\chapter*{Candidate Statement of Originality}

I hereby certify that the work embodied in this thesis is the result of original research, is free of plagiarised materials, and has not been submitted for a higher degree to any other University or Institution. I confirm that the investigations were conducted in accord with the ethics policies and integrity standards of Nanyang Technological University and that the research data are presented honestly and without prejudice.

\vspace{4cm}

\noindent
\begin{tabular*}{\textwidth}{%
  @{\extracolsep{\fill}}
  w{l}{3cm}
  c
  w{l}{6cm}
  @{}
}
% \today 
&& 
\includegraphics[width=5cm]{class/logos/ase_watermark.png} \\

\cmidrule(r){1-1}\cmidrule(l){3-3}
Date && Maricar L. Rabonza
\end{tabular*}

% \stackinset{r}{}{b}{}{\includegraphics[width=5cm]{maricar-esig.png}}{%
% \includegraphics[width=5cm]{class/logos/ase_watermark.png}}

% \begin{figure}
%   \centering
%   \includegraphics[width=.7\linewidth]{example-image-a}% Background/main image
%   \makebox[0pt][r]{% Similar to \llap
%     \raisebox{1em}{%
%       \includegraphics[width=.2\linewidth]{example-image-b}% Inserted image/inset
%     }\hspace*{1em}%
%   }%
%   \caption{An inset image}
% \end{figure}



% \makebox[0pt][r]{% Similar to \llap
% \raisebox{1em}{%
% \includegraphics[width=.2\linewidth]{maricar-esig}% Inserted image/inset
% }\hspace*{1em}%

%%%%%%%%%%%%%%%%%%%%%%%%%
\chapter*{Supervisor declaration}

I have reviewed the content and presentation style of this thesis and declare it is of sufficient quality and grammatical clarity to be examined.  To the best of my knowledge, it is free of plagiarism and the research and writing are those of the candidate except as acknowledged in the Author Attribution Statement. To the best of my knowledge, the investigations were conducted in accord with the ethics policies and integrity standards of Nanyang Technological University and that the research data are presented honestly and without prejudice.

\vspace{4cm}

\noindent
\begin{tabular*}{\textwidth}{%
  @{\extracolsep{\fill}}
  w{l}{3cm}
  c
  w{l}{6cm}
  @{}
}
% \today 
&& \includegraphics[width=5cm]{class/logos/ase_watermark.png} \\
\cmidrule(r){1-1}\cmidrule(l){3-3}
Date && Asst. Prof. David Lallemant
\end{tabular*}




%%%%%%%%%%%%%%%%%%%%%%%%%
\chapter*{Authorship Attribution Statement}

This thesis contains material from the following 5 papers where I was the first author.
\begin{itemize}
\setlength\itemsep{-0.5em}
\item 1 published journal article
\item 1 published contributing article to the United Nations Office for Disaster Risk Reduction (UNDRR) Global Assessment Report (GAR) on Disaster Risk Reduction 2022 
\item 1 journal article under review
\item 2 conference papers (peer-reviewed by 2 or more reviewers)
\end{itemize}
% Please amend the statements below to suit your circumstances if (B) is selected. Include all papers that are submitted, published, and in preparation.



%%%%%%%%%%%%%%%%%%%%%%%%%%% MAIN PAPERS
\vspace{.2cm}
\subsubsection*{Chapter 2:}
\\ \\ \noindent
Chapter \ref{chap-tephra} is a journal article under review:
\\ \\ \noindent
    \textbf{Rabonza M.L.}, Nguyen M., Biasse S., Jenkins S., Taisne B., Lallemant D., (2023) Inversion and forward estimation with process-based models: an investigation into cost functions, uncertainty-based weights and model-data fusion. \textit{Environmental Modelling & Software}, Under review.
\\ \\ \noindent
    The contributions of the co-authors are as follows:
    \begin{itemize}
    \setlength\itemsep{-0.45em}
    \item I lead the writing, research, and analysis.
    \item I designed the study with Michele Nguyen and David Lallemant. 
    \item Sebastien Biasse provided guidance in conducting tephra inversion modelling.
    \item All authors discussed the results and commented on the manuscript.
    \end{itemize}

%%%%%%%%%%%%%%%%%%%%%%%%%%%
\vspace{.2cm}
\subsubsection*{Chapter 3:}
\\ \\ \noindent
% 3500 words
Chapter \ref{chap-time} is a peer-reviewed conference paper published as follows.
\\ \\ \noindent
    \textbf{Rabonza, M.L.} and Lallemant, D. (2019). Accounting for time and state-dependent vulnerability of structural systems. \textit{In Proceedings of the 13th International Conference on Applications of Statistics and Probability in Civil Engineering (ICASP13) Sep 2019}. pp. 2298-2305. https://doi.org/10.22725/ICASP13.465
\\ \\ \noindent
    The contributions of the co-authors are as follows:
    \begin{itemize}
    \setlength\itemsep{-0.45em}
    \item I wrote the paper, collected the data, and performed the analysis.
    \item I designed the study with David Lallemant.
    \item David Lallemant commented on the manuscript.
    \end{itemize}

%%%%%%%%%%%%%%%%%%%%%%%%%%%
\vspace{.2cm}
\subsubsection*{Chapter 4:}
\\ \\ \noindent
Chapter \ref{chap-counterfactual} is a journal article published as:
\\ \\ \noindent
    \textbf{Rabonza M.L.}, Lin Y.C. and Lallemant D. (2022) Learning From Success, Not Catastrophe: Using Counterfactual Analysis to Highlight Successful Disaster Risk Reduction Interventions. \textit{Front. Earth Sci.} 10:847196. doi: 10.3389/feart.2022.847196
\\ \\ \noindent
    The contributions of the co-authors are as follows:
    \begin{itemize}
    \setlength\itemsep{-0.45em}
    \item I lead the writing, research, and analysis. 
    \item Yolanda Lin and David Lallemant provided feedback that shaped the analysis.
    \item David Lallemant conceived of the idea of celebrating successes in disaster risk reduction using counterfactual analysis. 
    \item All authors contributed to the conceptualisation and design of the study. 
    \end{itemize}

%%%%%%%%%%%%%%%%%%%%%%%%%%% ANNEX
\vspace{.2cm}
\subsubsection*{Appendix B}
% The Appendices of the thesis includes 2 papers where I was the first author. These papers are cited in Chapters 3 and 4 as Supplementary Material.
\\ \\ \noindent
Appendix \ref{app-time-paper} is a peer-reviewed conference paper, which is cited in Chapter \ref{chap-time}.
\\  \noindent \\
\textbf{Rabonza, M.L.} and Lallemant, D. (2018). A time-dependent model for seismic risk reduction policy analysis. Accepted in the \textit{17th U.S.-Japan-New Zealand Workshop on the Improvement of Structural Engineering and Resilience Nov 2018. https://hdl.handle.net/10356/164235}

\begin{itemize}
    \setlength\itemsep{-0.45em}
    \item I wrote the paper, collected the data, and performed the analysis.
    \item I designed the study with David Lallemant.
    \item David Lallemant commented on the manuscript.
    \end{itemize}


%%%%%%%%%%%%%%%%%%%%%%%%%%%%%%
\vspace{.2cm}
\subsubsection*{Appendix C}

\\ \\ 
\noindent
Appendix \ref{app-pls-2} is a paper published as a Contributing Paper to the Global Assessment Report (GAR) on Disaster Risk Reduction 2022 by the United Nations Office for Disaster Risk Reduction (UNDRR). The paper is cited in the Introduction and Chapter \ref{chap-counterfactual}.
% Cited in Chapter 4 is a paper published as a Contributing Paper to the Global Assessment Report (GAR) on Disaster Risk Reduction 2022 by the United Nations Office for Disaster Risk Reduction (UNDRR)
\\ \\
    \textbf{Rabonza, M.L.}, Lallemant,  D.,  Lin,  Y. C., Tadepalli,  S., Wagenaar,  D., Nguyen,  M., Choong,  J., Liu,  C. J. N., Sarica,  G. M., Widawati,  B. A. M., Balbi,  M., Khan,  F., Loos,  S. \& Lim,  T. N. (2022). Shedding light on avoided disasters : measuring the invisible benefits of disaster risk management using probabilistic counterfactual analysis. \textit{A contributing paper to the United Nations Office for Disaster Risk Reduction (UNDRR) Global Assessment Report 2022}. https://www.undrr.org/publication/shedding-light-avoided-disasters-measuring-invisible-benefits-disaster-risk-reduction
    \begin{itemize}
    \setlength\itemsep{-0.5em}
    \item I lead the writing, analysis design, and conceptualisation of the invisibilities in mitigation successes with David Lallemant.
    \item I designed the use of counterfactual analysis to address the invisibilities in risk reduction success with Yolanda Lin and David Lallemant.
    \item I produced the schematic figure for the types of invisibilities.
    \item For the risk perception section, Sanjana Tadepalli and Feroz Khan provided additional references, and assisted with editing.
    \item For the risk analysis framework section, Mariano Balbi and Yolanda Lin provided feedback on the writing.
    \item For the Nepal case study, I performed the analysis. Yolanda Lin and Sabine Loos provided feedback on the writing.
    \item For the India case study, Dennis Wagenaar and Bernadeti Ausie Miranda Widawati collected data and performed the analysis. Celine Liu contributed to the writing of the case study.
    \item For both case studies, Jeanette Choong produced the maps, and Michele Nguyen produced the visualisations of fatalities in the results.
    \item Michele Nguyen and Gizem Mestav Sarica assisted in collecting examples of disaster risk management (DRM) measures for which counterfactual analysis can be applied.
    \item Tian Ning Lim contributed to the writing about other metrics of success in DRM activities.
    \item All authors discussed the results and commented on the manuscript.
    
    \end{itemize}
    \\ \\ \noindent


\vspace{1 cm}
\\ \\ \noindent
We the undersigned agree with the above stated “proportion of work undertaken” for each of the above published (or submitted) peer-reviewed manuscripts contributing to the thesis:

 \vspace{.5cm}
\noindent Signed:
% \vspace{1cm}

\noindent 
\parbox[b]{0.4\linewidth}{% size of the first signature box
    \strut 
    \includegraphics[width=5cm]{class/logos/ase_watermark.png}
    \hrule
    \vspace{0.5cm}
    Maricar L. Rabonza \\
    Student\\
    Asian School of the Environment \\
    Nanyang Technological University \\
    Date: } 
\hspace{1cm} % distance between the two signature blocks % do not add par here
\parbox[b]{0.4\linewidth}{% ...and the second one
    \strut 
    \includegraphics[width=5cm]{class/logos/ase_watermark.png}
    \hrule
    \vspace{0.5cm}
    Asst. Prof. David Lallemant \\
    Supervisor \\
    Asian School of the Environment \\
    Nanyang Technological University \\
    Date: } 
    % \par\vspace{1cm} 
    
\vspace{1.5cm}
\noindent 
\parbox[b]{0.4\linewidth}{% size of the first signature box
    \strut 
    \includegraphics[width=5cm]{class/logos/ase_watermark.png}
    \hrule
    \vspace{0.5cm}
    Assoc Prof. Benoit Taisne \\
    Associate Chair (Research) \\
    Asian School of the Environment \\
    Nanyang Technological University \\
    Date: } 


}