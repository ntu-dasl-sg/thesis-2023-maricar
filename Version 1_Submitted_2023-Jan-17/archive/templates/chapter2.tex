\chapter{Language}

You are strongly encouraged to write your report in English, for two reasons. First, it will improve your use of English language. Second, it will increase visibility for you, the author, as well as for the Department of Computer Science, and for your host company (if any).

However, note that your examiner (and supervisors) are not there to provide you with extensive language feedback. We recommend that you check the language used in your report in several ways:
\begin{description}
\item[Reference books] dedicated to language issues can be very useful. \cite{heffernan2000writing} 
\item[Spelling and grammar checkers] which are usually available in the commonly used text editing environments.
\item[Colleagues and friends] willing to provide feedback your writing.
\item[Studieverkstaden] is a university level workshop, that can help you with language related problems (see \href{http://www.lu.se/studera/livet-som-student/service-och-stod/studieverkstaden}{Studieverkstaden}'s web page).
\item[Websites] useful for detecting language errors or strange expressions, such as
\begin{itemize}
\item \url{http://translate.google.com}
\item \url{http://www.gingersoftware.com/grammarcheck/}
\end{itemize}
\end{description}

\section{Style Elements}
Next, we will just give some rough guidelines for good style in a report written in English. Your supervisor and examiner as well as the aforementioned \textbf{Studieverkstad} might have a different  take on these, so we recommend you follow their advice whenever in doubt. If you want a reference to a short style guide, have a look at \cite{shortstyleguide}.

\subsubsection{Widows and Orphans}

Avoid \textit{widows} and \textit{orphans}, namely words or short lines at the beginning or end of a paragraph, which are left dangling at the top or bottom of a column, separated from the rest of the paragraph.

\subsubsection{Footnotes}

We strongly recommend you avoid footnotes. To quote from \cite{OGSW}, \textit{Footnotes are frequently misused by containing information which should either be placed in the text or excluded altogether. They should be avoided as a general rule and are acceptable only in exceptional cases when incorporation of their content in the text  [is] not possible.} 

\subsubsection{Active vs. Passive Voice}

Generally active voice (\textit{I ate this apple.}) is easier to understand than passive voice (\textit{This apple has been eaten (by me).}) In passive voice sentences the actor carrying out the action is often forgotten, which makes the reader wonder who actually performed the action. In a report is important to be clear about who carried out the work. Therefore we recommend to use active voice, and preferably the plural form \textit{we} instead of \textit{I} (even in single author reports).

\subsubsection{Long and Short Sentences}
A nice brief list of sentence problems and solutions is given in \cite{yalesentences}. Using choppy sentences (too short) is a common problem of many students. The opposite, using too long sentences, occurs less often, in our experience.

\subsubsection{Subject-Predicate Agreement}
A common problem of native Swedish speakers is getting the subject-predicate (verb) agreement right in sentences. Note that a verb must agree in person and number with its subject. As a rough tip, if you have subject ending in \textit{s} (plural), the predicate should not, and the other way around. Hence, \textit{only one s}. Examples follow:
\begin{description}
\item[incorrect] He have to take this road.
\item[correct] He has to take this road.
\end{description}
\begin{description}
\item[incorrect] These words forms a sentence.
\item[correct] These words form a sentence.
\end{description}
\noindent In more complex sentences, getting the agreement right is trickier. A brief guide is given in  the \textit{20 Rules of Subject Verb Agreement} \cite{subjectverb}.

